\begin{thebibliography}{9}


% --------------------KSIAZKI
% ------------------- rootkity

\bibitem{latexcompanion} 
Bill Blunden.
\textit{The Rootkit ARSENAL Escape and Evasion in the Dark Corners of the System Second Edition}.
\\ Jones \& Bartlett LEARNING 2013. ISBN: 978-1-4496-2636-5.

\bibitem{latexcompanion} 
Ric Vieler. 
\textit{Professional Rootkits}. 
\\ Wiley Publishing 2007. ISBN:978-0-47-010154-4.

\bibitem{latexcompanion} 
Greg Hoglund, Jamie Butler. 
\textit{Rootkity. Sabotowanie jądra systemu Windows}. 
\\Helion SA 2006. ISBN:83-246-0257-7.

\bibitem{latexcompanion} 
Filip Grzonkowski. 
\textit{Rootkity. Niewidzialny sabotaż elektroniczny}. 
\\ Wydawnictwo CSH. ISBN: 978-83-923745-7-2.


\bibitem{latexcompanion} 
Michael A. Davis, Sean M. Bodmer, Aaron LeMasters.
\textit{Hacking Exposed Malware \& Rootkits Secrets \& Solutions}. 
\\ The McGraw-Hill Companies 2010. ISBN: 978-0-07-159119-5.


% ------- złośliwe oprogramowanie - tworzenie
\bibitem{latexcompanion} 
Dawid Farbaniec. 
\textit{Techniki Twórców Złośliwego Oprogramowania Elementarz Programisty}. 
\\ Helion SA 2014. ISBN:978-83-246-8862-3.

\bibitem{latexcompanion} 
Dawid Farbaniec. 
\textit{Cyberwojna Metody działania hakerów}. 
\\ Helion SA 2018. ISBN: 978-83-283-4332-0.

\bibitem{latexcompanion} 
Chris Anley, John Heasman, Felix “FX” Linder, Gerardo Richarte. 
\textit{The Shellcoder’s Handbook Discovering and Exploiting Security Holes Second Edition}. 
\\ Helion 2013. ISBN:978-83-246-6653-9.


\bibitem{latexcompanion} 
Joxean Koret, Elias BachaaLany. 
\textit{The Antivirus Hacker’s Handbook}. 
\\ John Wiley \& Sons 2015. ISBN: 978-1-119-02875-8.

% -------------------wykrywanie rootkitow


\bibitem{latexcompanion} 
Michael Sikorski, Andrew Honig. 
\textit{Practical Malware Analysis The Hands-On Guide to Dissecting Malicious Software}. 
\\No Starch Press  2012.  ISBN: 1-59327-290-1.


\bibitem{latexcompanion} 
Michael Hale Ligh, Steven Adair, Blake Hartsein, Matthew Richard. 
\textit{Malware Amalyst’s Cookbook and DVD Tools and Techniques for Fighting Malicious Code}. 
\\ Wiley Publishing, Inc.2011.  ISBN: 978-0-470-61303-0.



% ---------------------- WINDOWS programowanie

\bibitem{latexcompanion} 
Charles Petzold. 
\textit{Programming Windows, 5th Edition}. 
\\ Microsoft Press 1998. ISBN: 157231995X.

\bibitem{latexcompanion} 
Johnson M. Hart. 
\textit{Windows System Programming 4 edition}. 
\\ Addison-Wesley Professional. ISBN:  0321657748.


\bibitem{latexcompanion} 
Pavel Yosifovich, Alex Ionescu, Mark E.Russinovich, David A. Solomon. 
\textit{Windows od środka. Architektura systemu, procesy, wątki, zarządzanie pamięcią i dużo więcej. Wydanie VII}. 
\\ Helion SA 2018. ISBN:978-83-283-3901-9.


\bibitem{latexcompanion} 
Rajeev Nagar. 
\textit{Windows NT file System Internals A Developer’s Guide 1 edition}. 
\\ O'Reilly Media 1997. ISBN: 1565922492.

\bibitem{latexcompanion} 
William Stallings. 
\textit{Systemy operacyjne. Architektura, funkcjonowanie i projektowanie. Wydanie IX}. 
\\ Helion 2018. ISBN:978-83-283-3760-2.


% -------wykrywanie/ pentesty 



\bibitem{latexcompanion} 
Patrick Engebretson. 
\textit{Hacking i testy penetracyjne. Podstawy}. 
\\ Helion 2013. ISBN:978-83-246-6653-9.


\bibitem{latexcompanion} 
Thomas Wilhelm. 
\textit{Profesjonalne testy penetracyjne. Zbuduj własne środowisko do testów}. 
\\ Helion 2014. ISBN:978-83-246-9036-7.


\bibitem{latexcompanion} 
Wil Allsopp. 
\textit{Testy Penetracyjne Dla Zaawansowanych}. 
\\ Helion SA 2018. ISBN:978-83-283-3895-1.

\bibitem{latexcompanion} 
Vijay Kumar Velu. 
\textit{Kali Linux Testy penetracyjne i bezpieczeństwo sieci dla zaawansowanych}. 
\\ Helion SA 2018. ISBN:978-83-283-4037-4.






% -------------bezpieczeństwo systemów ogólnie - zabezpieczenia
\bibitem{latexcompanion} 
Grillenmeier Guido, Clercq Jan. 
\textit{Bezpieczeństwo Microsoft Windows. Podstawy praktyczne}. 
\\Wydawnictwo Naukowe PWN 2008. ISBN: 978-8-3011-5307-6.


\bibitem{latexcompanion} 
William Stallings, Lawrie Brown. 
\textit{Bezpieczeństwo Systemów Informatycznych Zasady i praktyka Tom 1 Wydanie IV}. 
\\Helion SA 2019. ISBN: 978-83-283-4299-6.


\bibitem{latexcompanion} 
William Stallings, Lawrie Brown. 
\textit{Bezpieczeństwo Systemów Informatycznych Zasady i praktyka Tom 2 Wydanie IV}. 
\\ Helion SA 2019. ISBN: 978-83-283-4300-9.

\bibitem{latexcompanion} 
Lee Brotherston, Amanda Berlin. 
\textit{Bezpieczeństwo defensywne}. 
\\Helion SA 2019. ISBN:978-83-283-4722-9.

\bibitem{latexcompanion} 
Georgia Weidman. 
\textit{Bezpieczny system w praktyce Wyższa szkoła hackingu i testy penetracyjne}. 
\\ Helion SA 2015. ISBN:978-83-283-0352-2.

\bibitem{latexcompanion} 
Gene Kim, Jez Humble, Patrick Debois, John Willis.
\textit{DevOps Światowej klasy zwinność, Niezawodność i Bezpieczeństwo w twojej organizacji}. 
\\ Helion SA 2017. ISBN:978-83-283-3453-3.






% ----------------------- C programowanie
\bibitem{latexcompanion} 
Brian W. Kernighan, Dennis M. Ritchie.
\textit{Język ANSI C}. 
\\ Helion 2010. ISBN:978-83-246-2578-9.


\bibitem{latexcompanion} 
Richard Reese.
\textit{Wskaźniki w języku C}. 
\\Helion 2014. ISBN:978-83-246-8289-8.


\bibitem{latexcompanion} 
Jacek Galowicz.
\textit{C++ 17 STL Receptury}. 
\\ Helion 2018. ISBN: 978-83-283-4501-0.


\bibitem{latexcompanion} 
Jerzy Grębosz.
\textit{Opus magnum C++ 11. Programowanie w języku C++}. 
\\ Helion SA 2017. ISBN: 978-83-283-4214-9.

\bibitem{latexcompanion} 
Bruce Eckel.
\textit{Thinking in C++. Edycja polska}. 
\\ Helion SA 2002. ISBN: 83-7197-709-3.


\bibitem{latexcompanion} % meta do zmian w locie
David Abrahams, Aleksey Gurtovoy.
\textit{Język C++. Metaprogramowanie za pomocą szablonów}. 
\\ Helion SA 2005. ISBN: 83-7361-935-6.


\bibitem{latexcompanion}
Bjarne Stroustrup.
\textit{Programowanie. Teoria i praktyka z wykorzystaniem C++. Wydanie II poprawione}. 
\\ Helion SA 2013. ISBN: 978-83-246-7720-7.


\bibitem{latexcompanion}
Stephan Roth.
\textit{Czysty kod w C++ 17 Oprogramowanie Łatwe w Utrzymaniu}. 
\\Helion 2018. ISBN: 978-83-283-4340-55.

% -----------------------Assembler

\bibitem{latexcompanion} 
Kip R. Irvine.
\textit{Asembler dla procesorów Intel. Vademecum profesjonalisty}. 
\\ Helion SA 2003. ISBN: 83-7197-910-X.


\bibitem{latexcompanion} 
Alexey Lyashko.
\textit{Mastering Assembly Programming}. 
\\Pack Publishing 2017. ISBN: 9781787120075.



% --------------------ARTYKULY/TUTORIALE
\bibitem{linczelion} 
Karol Celebi, Zbigniew Suski.
\textit{Rootkit dla dydaktycznego systemu Linux}. 
Przegląd Teleinformatyczny NR 1-2, 2016.



% --------------------------------------------------------
% ------------------ LINKI
\bibitem{Malware} 
Reiner Creutzburg, Simon Heglang.
\\\texttt{https://www.researchgate.net/profile/Reiner\_Creutzburg/publication/305469492\_Handbook\_of\_Malware\_2016\_-\_A\_Wikipedia\_Book/links/578fe89e08ae64311c0c79b8/Handbook-of-Malware-2016-A-Wikipedia-Book.pdf}


\bibitem{desktopWin} 
mikeblome, olprod. Przewodnik: Tworzenie tradycyjnych aplikacji Windows Desktop (C++)
\\\texttt{https://docs.microsoft.com/pl-pl/cpp/windows/walkthrough-creating-windows-desktop-applications-cpp?view=vs-2019}


\bibitem{linczelion} 
linczelion.Win32 Assembly Tutorials.
\\\texttt{http://win32assembly.programminghorizon.com/tutorials.html}


\bibitem{linczelion} 
Chris Lomont. Introduction to x64 Assembly
\\\texttt{https://software.intel.com/en-us/articles/introduction-to-x64-assembly}


\bibitem{linczelion} 
Beth Levin, Andres Mariano Gorzelany,Justin Hall, Nick Schonning, Duncan Mackenzie. Understanding malware \& other threats
\\\texttt{https://docs.microsoft.com/pl-pl/windows/security/threat-protection/intelligence/understanding-malware\#shellcode}


\bibitem{linczelion} 
Adrian “Vizzdoom” Michalczyk. Jak w prosty sposób oszukać ochronę antywirusową
\\\texttt{https://sekurak.pl/jak-w-prosty-sposob-oszukac-ochrone-antywirusowa-av-bypassing/}

\bibitem{linczelion} 
afreak. BIOS Based Rootkits
\\\texttt{http://www.exfiltrated.com/research-BIOS\_Based\_Rootkits.php}

\bibitem{linczelion} 
Kod źródłowy rootkita PSF
\\\texttt{https://repo.palkeo.com/repositories/mirror7.meh.../Rootkit/psf.c}




%------------------ KONIEC Bibliografii -------------

\end{thebibliography}